\documentclass{beamer}
\newcommand{\tituloaula}{\'{A}rvore Bin\'aria Aleat\'oria de Busca}
\usepackage{disciplina}

\newcommand{\figlabel}[1]{\label{fig:#1}}
\newcommand{\eq}{=}
%\usetheme{CambridgeUS}         % tema
% \usecolortheme{orchid}      % cores
\usetheme{metropolis}
\metroset{numbering=fraction, sectionpage=none, subsectionpage=progressbar}
\usecolortheme{seahorse}
\usecolortheme{rose}
\usefonttheme[onlysmall]{structurebold}
\usefonttheme[onlymath]{serif} % fonte modo matematico

\usepackage{framed}
\usepackage{disciplina}
\usepackage{tikz}
\usetikzlibrary{calc,shapes.multipart,chains,arrows}
\usepgflibrary{shapes.multipart}
\newcommand{\eq}{=}

\author[João Araujo Ribeiro]{João Araujo Ribeiro \\ \texttt{jaraujo@uerj.br}}
\institute[UERJ]{Universidade do Estado do Rio de Janeiro} % opcional
\date[EstrInf]{Departamento de Engenharia de Sistemas e Computação}
\pgfdeclareimage[height=0.7cm]{imagens//logo.png}{imagens//logodesc.png}
\logo{\pgfuseimage{imagens//logo.png}}
% Titulo
\title[\sc{Estruturas de Informação I}]{Estruturas de Informação I - Aula 18}
\subtitle{\tituloaula}
 
\begin{document}

\begin{frame}
  \titlepage 
\end{frame}

\begin{frame}
	\frametitle{Resumo}
	\tableofcontents
\end{frame}

\section{EI14 -  \tituloaula}
\begin{frame}
	\LARGE{\alert{\tituloaula}}
	\normalsize
	
Uma árvore binária de busca usa a aleatoriedade para conseguir $O(\log n)$ de tempo esperado para todas as operações.	
	
\end{frame}

\begin{frame}
\frametitle{Considere as duas árvores binárias de busca mostradas abaixo, de quantas maneiras diferentes podemos montá-las?}
\begin{figure}
  \begin{center}
    \begin{tabular}{cc}
      \includegraphics[scale=0.90909,scale=0.95]{../Book-datastructures/ods/latex/figs/bst-path} &
      \includegraphics[scale=0.90909,scale=0.95]{../Book-datastructures/ods/latex/figs/bst-balanced}
    \end{tabular}
  \end{center}
  \caption{Duas árvores binárias de busca com inteiros $0,\ldots,14$.}
  \figlabel{rbs-lvc}
\end{figure}
\end{frame}
\begin{frame}
\frametitle{}
A árvore da esquerda ocorre se começamos com uma \textit{Árvore Binária de Busca} vazia e acrescentamos a sequência:
\[
    \langle 0,1,2,3,4,5,6,7,8,9,10,11,12,13,14 \rangle \enspace .
\]
\begin{figure}
  \begin{center}
      \includegraphics[scale=0.90909,scale=0.95]{../Book-datastructures/ods/latex/figs/bst-path} 
  \end{center}
  \caption{Árvore binária de busca com inteiros $0,\ldots,14$.}
\end{figure}
\end{frame}


\begin{frame}
\frametitle{Árvore de Busca}
A árvore da direita pode ser criada pela sequência:
\[
    \langle 7,3,11,1,5,9,13,0,2,4,6,8,10,12,14 \rangle  \enspace .
\]
\begin{figure}
  \begin{center}
      \includegraphics[scale=0.90909,scale=0.95]{../Book-datastructures/ods/latex/figs/bst-balanced}
  \end{center}
  \caption{Árvore binária de busca com inteiros $0,\ldots,14$.}
  \figlabel{rbs-lvc}
\end{figure}
\end{frame}

\begin{frame}
ou
\[
    \langle 7,3,1,5,0,2,4,6,11,9,13,8,10,12,14 \rangle  \enspace ,
\]
\begin{figure}
  \begin{center}
      \includegraphics[scale=0.90909,scale=0.95]{../Book-datastructures/ods/latex/figs/bst-balanced}
  \end{center}
  \caption{Árvore binária de busca com inteiros $0,\ldots,14$.}
  \figlabel{rbs-lvc}
\end{figure}
\end{frame}

\begin{frame}

ou
\[
    \langle 7,3,1,11,5,0,2,4,6,9,13,8,10,12,14 \rangle \enspace .
\] 
\begin{figure}
  \begin{center}
      \includegraphics[scale=0.90909,scale=0.95]{../Book-datastructures/ods/latex/figs/bst-balanced}
  \end{center}
  \caption{Árvore binária de busca com inteiros $0,\ldots,14$.}
  \figlabel{rbs-lvc}
\end{figure}

\pause
De fato, existem $21,964,800$ sequências que geram a árvore da direita e somente uma que gera a árvore da esquerda.

\end{frame}
\begin{frame}
$ \#modos = $ 

( $\#$intercalações entre $S_L$ e $S_R$)X($\#$Possíveis $S_L$)X($\#$Possíveis $S_R$)
\end{frame}
\begin{frame}
\frametitle{Mais provável}
O exemplo nos dá uma evidência factual que, se escolhemos
uma permutação aleatória de 0, . . . , 14, e inserimos em uma árvore binária,
é mais provável obtermos uma árvore balanceada que obtermos uma árvore totalmente desbalanceada.
\end{frame}

\begin{frame}
\frametitle{}

Uma \emph{árvore binária aleatória de busca}
de tamanho $n$ é obtida do seguinte modo:
  
Considere uma permutação aleatória, $x_0,\ldots,x_{n-1}$,
de inteiros $0,\ldots,n-1$ e insira seus elementos, um a um,
em uma ArvoreBinariaDeBusca.
\end{frame}

\begin{frame}
\frametitle{Permutação aleatória}
Por \emph{permutação aleatória}
queremos dizer
que cada possível $n!$ permutação (ordenação) de $0,\ldots,n-1$
é igualmente provável, assim como que a probabilidade de obter qualquer permutação particular é $1/n!$.
\end{frame}

\begin{frame}
\frametitle{Busca}

Uma árvore binária aleatória de busca pode ser construída em um tempo $O(n\log n)$.
Em uma árvore binária aleatória de busca, a operação $find(x)$ tem um tempo esperado de $O(\log n)$.

\end{frame}

\begin{frame}
\frametitle{$Treap$: Uma árvore Binária de Busca Aleatorizada}
O problema com as árvores binárias aleatórias de buscas é, é claro, que elas não são
dinâmicas.  Elas não suportam as operações $add(x)$ ou $remove(x)$
necessárias

\end{frame}

\begin{frame}
\frametitle{Prioridade}
Um nó em uma $Treap$ é como um nó em uma $ArvoreBinariaDeBusca$ no sentido que
ele tem um valor para o dado, $x$, porém ele também tem uma \emph{prioridade},
numérica única, $p$, que é associada aleatoriamente:
\end{frame}

\begin{frame}
Adicionalmente, para ser um árvore binária de busca, os nós de uma $Treap$
também obedecem à \emph{propriedade de heap}:
\begin{itemize}
\item (Propriedade de Heap)  Para cada nó $u$, exceto a raiz, 
      $u.pai.p < u.p$.
\end{itemize}
em outras palavras, cada nó possui uma prioridade menor que aquelas de seus dois filhos.

\end{frame}

\begin{frame}
\frametitle{Nó com prioridade}
\begin{figure}
  \begin{center}
    \includegraphics{../Book-datastructures/ods/latex/figs/treap}
  \end{center}
  \caption[Uma Treap]{Um exemplo de uma $Treap$ contendo os inteiros
  $0,\ldots,9$. Cada nó, $u$, é ilustrado como uma caixa contendo $u.x,u.p$.}
  \figlabel{treap}
\end{figure}

\end{frame}

\begin{frame}
\frametitle{heap e busca}
As condições de heap e de árvore binária de busca juntas garantem que, uma vez
que a chave ($x$) e a prioridade ($p$) de cada nó seja definido, o formato
da $Treap$ está completamente determinado. 
\end{frame}

\begin{frame}
\frametitle{Propriedade de heap}
A propriedade de heap nos diz
que o nó com prioridade mínima tem que ser a raiz, $r$, da $Treap$.
A propriedade da árvore binária de busca nos diz que todos os nós com chaves menores
que $r.x$ são armazenados na subárvore com raiz em $r.esquerdo$ e todos os nós com
chaves maiores que $r.x$ são armazenados na subárvore com raiz em $r.direito$.
\end{frame}

\begin{frame}
\frametitle{}
Um ponto importante sobre os valores de prioridade em uma $Treap$ é que eles
são únicos e atribuídos aleatoriamente.  Por conta disso, existem dois modos
equivalentes de pensar sobre uma $Treap$.  Como definido acima, uma
$Treap$ obedece às propriedades do heap e da árvore binária de busca.  Alternativamente,
podemos pensar uma $Treap$ como uma $ArvoreBinariaDeBusca$ cujos nós
sejam inseridos em uma ordem crescente de prioridade.
\end{frame}

\begin{frame}
\frametitle{}
A $Treap$ abaixo pode ser obtida com a inserção da sequência de valores $(x,p)$
\[
  \langle
   (3,1), (1,6), (0,9), (5,11), (4,14), (9,17), (7,22), (6,42), (8,49), (2,99)
  \rangle
\]
\begin{figure}
  \begin{center}
    \includegraphics{../Book-datastructures/ods/latex/figs/treap}
  \end{center}
  \caption[Uma Treap]{Um exemplo de uma $Treap$ contendo os inteiros
  $0,\ldots,9$.}
  \figlabel{treap}
\end{figure}
\end{frame}

\begin{frame}
\frametitle{Operações add e delete}
O benefício real de uma $Treap$ é que ela pode 
suportar as operações $add(x)$ e $delete(x)$.  Para
fazer isto, ela precisa executar rotações de modo a manter 
a propriedade de heap. 
\end{frame}

\begin{frame}
\frametitle{Rotação}
Uma \emph{rotação}
\index{rotação}%
em uma árvore binária de busca
é uma modificação local que pega um pai $u$ de um nó $w$
e torna $w$ o pai de $u$, enquanto preserva a propriedade da árvore binária de busca.
 Rotações vêm com dois sabores: \emph{à esquerda} or \emph{à direita}
dependendo se $w$ é um filho direito ou esquerdo de $u$, respectivamente.
\end{frame}

\begin{frame}[shrink]
\frametitle{Rotações}
\begin{figure}
  \begin{center}
     \includegraphics[width=\textwidth]{../Book-datastructures/ods/latex/figs/rotation}
  \end{center}
  \caption{Rotações à esquerda e à direita em uma árvore binária de busca.}
  \figlabel{rotations}
\end{figure}
\end{frame}

\begin{frame}
\frametitle{}
O código que implementa isto deve prever essas duas possibilidades e
tomar cuidado com um caso limite (quando $u$ é a raiz), deste modo, o código real
é um pouco mais longo que a figura poderia levar um leitor a crer:
\end{frame}

\begin{frame}[shrink]
\frametitle{Rotate left}
\begin{flushleft}
\hspace*{1em}def \ensuremath{\mathrm{rotate\_left}(\ensuremath{\mathit{u}})}\\
\hspace*{1em} \hspace*{1em} \ensuremath{\ensuremath{\mathit{w}} \gets  \ensuremath{\ensuremath{\mathit{u}}.right}}\\
\hspace*{1em} \hspace*{1em} \ensuremath{\ensuremath{\mathit{w}}.\ensuremath{parent} \gets  \ensuremath{\ensuremath{\mathit{u}}.parent}}\\
\hspace*{1em} \hspace*{1em} {\color{black} \textbf{if}} \ensuremath{\ensuremath{\mathit{w}}.\ensuremath{\mathit{parent}} \ne nil} {\color{black} \textbf{then}} \\
\hspace*{1em} \hspace*{1em} \hspace*{1em} {\color{black} \textbf{if}} \ensuremath{\ensuremath{\mathit{w}}.\ensuremath{\mathit{parent}}.\ensuremath{\mathit{left}} \eq u} {\color{black} \textbf{then}} \\
\hspace*{1em} \hspace*{1em} \hspace*{1em} \hspace*{1em} \ensuremath{\ensuremath{\mathit{w}}.\ensuremath{\ensuremath{\mathit{parent}}.left} \gets  \ensuremath{w}}\\
\hspace*{1em} \hspace*{1em} \hspace*{1em} {\color{black} \textbf{else}} \\
\hspace*{1em} \hspace*{1em} \hspace*{1em} \hspace*{1em} \ensuremath{\ensuremath{\mathit{w}}.\ensuremath{\ensuremath{\mathit{parent}}.right} \gets  \ensuremath{w}}\\
\hspace*{1em} \hspace*{1em} \ensuremath{\ensuremath{\mathit{u}}.\ensuremath{right} \gets  \ensuremath{\ensuremath{\mathit{w}}.left}}\\
\hspace*{1em} \hspace*{1em} {\color{black} \textbf{if}} \ensuremath{\ensuremath{\mathit{u}}.\ensuremath{\mathit{right}} \ne nil} {\color{black} \textbf{then}} \\
\hspace*{1em} \hspace*{1em} \hspace*{1em} \ensuremath{\ensuremath{\mathit{u}}.\ensuremath{\ensuremath{\mathit{right}}.parent} \gets  \ensuremath{u}}\\
\hspace*{1em} \hspace*{1em} \ensuremath{\ensuremath{\mathit{u}}.\ensuremath{parent} \gets  \ensuremath{w}}\\
\hspace*{1em} \hspace*{1em} \ensuremath{\ensuremath{\mathit{w}}.\ensuremath{left} \gets  \ensuremath{u}}\\
\hspace*{1em} \hspace*{1em} {\color{black} \textbf{if}} \ensuremath{\ensuremath{\mathit{u}} \eq r} {\color{black} \textbf{then}}  \\
\hspace*{1em} \hspace*{1em} \hspace*{1em} \ensuremath{\ensuremath{\mathit{r}} \gets  \ensuremath{w}}\\
\hspace*{1em} \hspace*{1em} \hspace*{1em} \ensuremath{\ensuremath{\mathit{r}}.\ensuremath{parent} \gets  \ensuremath{nil}}\\
\end{flushleft}

\end{frame}

\begin{frame}[shrink]
\frametitle{Rotate Right}
\begin{flushleft}
\hspace*{1em} \ensuremath{\mathrm{rotate\_right}(\ensuremath{\mathit{u}})}\\
\hspace*{1em} \hspace*{1em} \ensuremath{\ensuremath{\mathit{w}} \gets  \ensuremath{\ensuremath{\mathit{u}}.left}}\\
\hspace*{1em} \hspace*{1em} \ensuremath{\ensuremath{\mathit{w}}.\ensuremath{parent} \gets  \ensuremath{\ensuremath{\mathit{u}}.parent}}\\
\hspace*{1em} \hspace*{1em} {\color{black} \textbf{if}} \ensuremath{\ensuremath{\mathit{w}}.\ensuremath{\mathit{parent}} \ne nil} {\color{black} \textbf{then}} \\
\hspace*{1em} \hspace*{1em} \hspace*{1em} {\color{black} \textbf{if}} \ensuremath{\ensuremath{\mathit{w}}.\ensuremath{\mathit{parent}}.\ensuremath{\mathit{left}} \eq u} {\color{black} \textbf{then}} \\
\hspace*{1em} \hspace*{1em} \hspace*{1em} \hspace*{1em} \ensuremath{\ensuremath{\mathit{w}}.\ensuremath{\ensuremath{\mathit{parent}}.left} \gets  \ensuremath{w}}\\
\hspace*{1em} \hspace*{1em} \hspace*{1em} {\color{black} \textbf{else}} \\
\hspace*{1em} \hspace*{1em} \hspace*{1em} \hspace*{1em} \ensuremath{\ensuremath{\mathit{w}}.\ensuremath{\ensuremath{\mathit{parent}}.right} \gets  \ensuremath{w}}\\
\hspace*{1em} \hspace*{1em} \ensuremath{\ensuremath{\mathit{u}}.\ensuremath{left} \gets  \ensuremath{\ensuremath{\mathit{w}}.right}}\\
\hspace*{1em} \hspace*{1em} {\color{black} \textbf{if}} \ensuremath{\ensuremath{\mathit{u}}.\ensuremath{\mathit{left}} \ne nil} {\color{black} \textbf{then}} \\
\hspace*{1em} \hspace*{1em} \hspace*{1em} \ensuremath{\ensuremath{\mathit{u}}.\ensuremath{\ensuremath{\mathit{left}}.parent} \gets  \ensuremath{u}}\\
\hspace*{1em} \hspace*{1em} \ensuremath{\ensuremath{\mathit{u}}.\ensuremath{parent} \gets  \ensuremath{w}}\\
\hspace*{1em} \hspace*{1em} \ensuremath{\ensuremath{\mathit{w}}.\ensuremath{right} \gets  \ensuremath{u}}\\
\hspace*{1em} \hspace*{1em} {\color{black} \textbf{if}} \ensuremath{\ensuremath{\mathit{u}} \eq r} {\color{black} \textbf{then}} \\
\hspace*{1em} \hspace*{1em} \hspace*{1em} \ensuremath{\ensuremath{\mathit{r}} \gets  \ensuremath{w}}\\
\hspace*{1em} \hspace*{1em} \hspace*{1em} \ensuremath{\ensuremath{\mathit{r}}.\ensuremath{parent} \gets  \ensuremath{nil}}\\
\end{flushleft}

\end{frame}

\begin{frame}
\frametitle{Profundidade de u}
Em termos da estrutura de dados Treap, a propriedade mais importante de uma
rotação é que a profundidade de \ensuremath{\ensuremath{\ensuremath{\mathit{w}}}} diminui de um enquanto a profundidade de \ensuremath{\ensuremath{\ensuremath{\mathit{u}}}}
aumenta de um. 
\end{frame}

\begin{frame}
\frametitle{Usando rotações, podemos implementar a operação \ensuremath{\ensuremath{\mathrm{add}(\ensuremath{\mathit{x}})}} como se segue:}

Criamos um novo nó, \ensuremath{\ensuremath{\ensuremath{\mathit{u}}}}, atribuímos \ensuremath{\ensuremath{\ensuremath{\mathit{u}}.\ensuremath{x}\gets \ensuremath{x}}}, e escolhemos um valor aleatório
para \ensuremath{\ensuremath{\ensuremath{\mathit{u}}.\ensuremath{\mathit{p}}}}.  Em seguida, inserimos \ensuremath{\ensuremath{\ensuremath{\mathit{u}}}} usando o algoritmo usual \ensuremath{\ensuremath{\mathrm{add}(\ensuremath{\mathit{x}})}} 
para uma ArvoreBinariaDeBusca, assim, \ensuremath{\ensuremath{\ensuremath{\mathit{u}}}} agora é uma nova folha de Treap.
Neste ponto, nossa Treap satisfaz a propriedade da árvore binária de busca,
mas não necessariamente a propriedade de heap.  Particularmente, pode ser 
o caso em que \ensuremath{\ensuremath{\ensuremath{\mathit{u}}.\ensuremath{\mathit{pai}}.\ensuremath{\mathit{p}} > \ensuremath{\mathit{u}}.\ensuremath{\mathit{p}}}}.  Se este é o caso, então executamos uma
rotação no nó \ensuremath{\ensuremath{\ensuremath{\mathit{w}}}}=\ensuremath{\ensuremath{\ensuremath{\mathit{u}}.\ensuremath{\mathit{pai}}}} de modo que \ensuremath{\ensuremath{\ensuremath{\mathit{u}}}} se torne o pai de \ensuremath{\ensuremath{\ensuremath{\mathit{w}}}}.
Se \ensuremath{\ensuremath{\ensuremath{\mathit{u}}}} continua a violar a propriedade de heap, teremos que repetir isso, 
diminuindo a profundidade de \ensuremath{\ensuremath{\ensuremath{\mathit{u}}}} por um a cada vez, até
que \ensuremath{\ensuremath{\ensuremath{\mathit{u}}}} se torne a raiz ou $\ensuremath{\ensuremath{\ensuremath{\mathit{u}}.\ensuremath{\mathit{pai}}.\ensuremath{\mathit{p}}}} < \ensuremath{\ensuremath{\ensuremath{\mathit{u}}.\ensuremath{\mathit{p}}}}$.

\end{frame}

\begin{frame}[shrink]
\frametitle{Add}
\begin{flushleft}
\hspace*{1em} \ensuremath{\mathrm{add}(\ensuremath{\mathit{x}})}\\
\hspace*{1em} \hspace*{1em} \ensuremath{\ensuremath{\mathit{u}} \gets  \ensuremath{\mathrm{new\_node}(\ensuremath{\mathit{x}})}}\\
\hspace*{1em} \hspace*{1em} {\color{black} \textbf{if}} \ensuremath{\mathrm{add\_node}(\ensuremath{\mathit{u}})} {\color{black} \textbf{then}} \\
\hspace*{1em} \hspace*{1em} \hspace*{1em} \ensuremath{\mathrm{bubble\_up}(\ensuremath{\mathit{u}})}\\
\hspace*{1em} \hspace*{1em} \hspace*{1em} {\color{black} \textbf{return}} \ensuremath{\ensuremath{\mathit{true}}}\\
\hspace*{1em} \hspace*{1em} {\color{black} \textbf{return}} \ensuremath{\ensuremath{\mathit{false}}}\\
\ \\
\hspace*{1em} \ensuremath{\mathrm{bubble\_up}(\ensuremath{\mathit{u}})}\\
\hspace*{1em} \hspace*{1em} {\color{black} \textbf{while}} \ensuremath{\ensuremath{\mathit{u}} \ne r} {\color{black} \textbf{and}} \ensuremath{\ensuremath{\mathit{u}}.\ensuremath{\mathit{parent}}.\ensuremath{\mathit{p}} > \ensuremath{\mathit{u}}.p} {\color{black} \textbf{do}} \\
\hspace*{1em} \hspace*{1em} \hspace*{1em} {\color{black} \textbf{if}} \ensuremath{\ensuremath{\mathit{u}}.\ensuremath{\mathit{parent}}.\ensuremath{\mathit{right}} \eq u} {\color{black} \textbf{then}} \\
\hspace*{1em} \hspace*{1em} \hspace*{1em} \hspace*{1em} \ensuremath{\mathrm{rotate\_left}(\ensuremath{\mathit{u}}.\ensuremath{\mathit{parent}})}\\
\hspace*{1em} \hspace*{1em} \hspace*{1em} {\color{black} \textbf{else}} \\
\hspace*{1em} \hspace*{1em} \hspace*{1em} \hspace*{1em} \ensuremath{\mathrm{rotate\_right}(\ensuremath{\mathit{u}}.\ensuremath{\mathit{parent}})}\\
\hspace*{1em} \hspace*{1em} {\color{black} \textbf{if}} \ensuremath{\ensuremath{\mathit{u}}.\ensuremath{\mathit{parent}} \eq nil} {\color{black} \textbf{then}} \\
\hspace*{1em} \hspace*{1em} \hspace*{1em} \ensuremath{\ensuremath{\mathit{r}} \gets  \ensuremath{u}}\\
\end{flushleft}
 
\end{frame}

\begin{frame}[shrink]
\frametitle{Exemplo}
\begin{figure}
  \begin{center}
  \includegraphics{../Book-datastructures/ods/latex/figs-python/treap-insert-a} \\
  \end{center}
  \caption[Inserindo em uma Treap]{Inserindo o valor 1.5 na Treap.}
  \figlabel{treap-add}
\end{figure}
\end{frame}

\begin{frame}[shrink]
\frametitle{Exemplo}
\begin{figure}
  \begin{center}
  \includegraphics{../Book-datastructures/ods/latex/figs-python/treap-insert-b} \\
  \end{center}
  \caption[Inserindo em uma Treap]{Inserindo o valor 1.5 na Treap.}
  \figlabel{treap-add}
\end{figure}
\end{frame}

\begin{frame}[shrink]
\frametitle{Exemplo}
\begin{figure}
  \begin{center}
  \includegraphics{../Book-datastructures/ods/latex/figs-python/treap-insert-c} \\
  \end{center}
  \caption[Inserindo em uma Treap]{Inserindo o valor 1.5 na Treap.}
  \figlabel{treap-add}
\end{figure}
\end{frame}

\begin{frame}
\frametitle{Tempo de execução de add(x)}
O tempo de execução de uma operação \ensuremath{\ensuremath{\mathrm{add}(\ensuremath{\mathit{x}})}} é dado pelo tempo 
que leva para seguir o caminho de busca para \ensuremath{\ensuremath{\ensuremath{\mathit{x}}}} mais o número de rotações
executadas para mover o nó recém adicionado, \ensuremath{\ensuremath{\ensuremath{\mathit{u}}}}, na sua localização correta
na Treap. 
\end{frame}

\begin{frame}
\frametitle{Comprimento esperado do caminho de busca}
o comprimento esperado do
caminho de busca é no máximo $2\ln \ensuremath{\ensuremath{\ensuremath{\mathit{n}}}}+O(1)$.  Além disso, cada rotação
diminui a profundidade de \ensuremath{\ensuremath{\ensuremath{\mathit{u}}}}.   Esse processo para se \ensuremath{\ensuremath{\ensuremath{\mathit{u}}}} se torna a raiz, assim
o número esperado de rotações não pode ultrapassar o comprimento esperado
do caminho de busca.  Então, o tempo esperado de execução da operação \ensuremath{\ensuremath{\mathrm{add}(\ensuremath{\mathit{x}})}}
em uma Treap é $O(\log \ensuremath{\ensuremath{\ensuremath{\mathit{n}}}})$. 
\end{frame}
\begin{frame}
\frametitle{Remove(x)}
A operação \ensuremath{\ensuremath{\mathrm{remove}(\ensuremath{\mathit{x}})}} em uma Treap é o oposto da operação \ensuremath{\ensuremath{\mathrm{add}(\ensuremath{\mathit{x}})}}.
Procuramos pelo nó, \ensuremath{\ensuremath{\ensuremath{\mathit{u}}}}, contendo \ensuremath{\ensuremath{\ensuremath{\mathit{x}}}}, então executamos
rotações para mover \ensuremath{\ensuremath{\ensuremath{\mathit{u}}}} para baixo até que ele se torne uma folha, e então separamos
\ensuremath{\ensuremath{\ensuremath{\mathit{u}}}} da Treap.
\end{frame}

\begin{frame}
\frametitle{}
Para mover \ensuremath{\ensuremath{\ensuremath{\mathit{u}}}} para baixo, podemos
executar ou uma rotação para esquerda ou uma pra direita em \ensuremath{\ensuremath{\ensuremath{\mathit{u}}}}, que vai substituir \ensuremath{\ensuremath{\ensuremath{\mathit{u}}}}
por \ensuremath{\ensuremath{\ensuremath{\mathit{u}}.\ensuremath{\mathit{direito}}}} ou \ensuremath{\ensuremath{\ensuremath{\mathit{u}}.\ensuremath{\mathit{esquerdo}}}}, respectivamente.
\end{frame}
\begin{frame}
\frametitle{Escolha}
A escolha é feita de acordo com a primeira situação seguinte que aparece:
\begin{enumerate}
\item Se \ensuremath{\ensuremath{\ensuremath{\mathit{u}}.\ensuremath{\mathit{esquerdo}}}} e \ensuremath{\ensuremath{\ensuremath{\mathit{u}}.\ensuremath{\mathit{direito}}}} são ambos \ensuremath{\ensuremath{\ensuremath{\mathit{nil}}}}, então \ensuremath{\ensuremath{\ensuremath{\mathit{u}}}} é uma folha e nenhuma rotação é feita.
\item Se \ensuremath{\ensuremath{\ensuremath{\mathit{u}}.\ensuremath{\mathit{esquerdo}}}} (ou \ensuremath{\ensuremath{\ensuremath{\mathit{u}}.\ensuremath{\mathit{direito}}}}) é \ensuremath{\ensuremath{\ensuremath{\mathit{nil}}}}, então executamos uma rotação à direita (ou à esquerda, respectivamente) em \ensuremath{\ensuremath{\ensuremath{\mathit{u}}}}.
\item Se $\ensuremath{\ensuremath{\ensuremath{\mathit{u}}.\ensuremath{\mathit{esquerdo}}.\ensuremath{\mathit{p}}}} < \ensuremath{\ensuremath{\ensuremath{\mathit{u}}.\ensuremath{\mathit{direito}}.\ensuremath{\mathit{p}}}}$ (ou $\ensuremath{\ensuremath{\ensuremath{\mathit{u}}.\ensuremath{\mathit{esquerdo}}.\ensuremath{\mathit{p}}}} > \ensuremath{\ensuremath{\ensuremath{\mathit{u}}.\ensuremath{\mathit{direito}}.\ensuremath{\mathit{p}}}})$, então executamos uma rotação à direita (ou rotação à esquerda, respectivamente) em \ensuremath{\ensuremath{\ensuremath{\mathit{u}}}}.
\end{enumerate}
Essas três regras asseguram que a Treap não se torne desconectada e que a 
propriedade de heap seja restabelecida uma vez que \ensuremath{\ensuremath{\ensuremath{\mathit{u}}}} seja removido.
\end{frame}

\begin{frame}[shrink]
\frametitle{Remove(x)}
\begin{flushleft}
\hspace*{1em} \ensuremath{\mathrm{remove}(\ensuremath{\mathit{x}})}\\
\hspace*{1em} \hspace*{1em} \ensuremath{\ensuremath{\mathit{u}} \gets  \ensuremath{\mathrm{find\_last}(\ensuremath{\mathit{x}})}}\\
\hspace*{1em} \hspace*{1em} {\color{black} \textbf{if}} \ensuremath{\ensuremath{\mathit{u}} \ne nil} {\color{black} \textbf{and}} \ensuremath{\ensuremath{\mathit{u}}.\ensuremath{\mathit{x}} \eq x} {\color{black} \textbf{then}} \\
\hspace*{1em} \hspace*{1em} \hspace*{1em} \ensuremath{\mathrm{trickle\_down}(\ensuremath{\mathit{u}})}\\
\hspace*{1em} \hspace*{1em} \hspace*{1em} \ensuremath{\mathrm{splice}(\ensuremath{\mathit{u}})}\\
\hspace*{1em} \hspace*{1em} \hspace*{1em} {\color{black} \textbf{return}} \ensuremath{\ensuremath{\mathit{true}}}\\
\hspace*{1em} \hspace*{1em} {\color{black} \textbf{return}} \ensuremath{\ensuremath{\mathit{false}}}\\
\ \\
\hspace*{1em} \ensuremath{\mathrm{trickle\_down}(\ensuremath{\mathit{u}})}\\
\hspace*{1em} \hspace*{1em} {\color{black} \textbf{while}} \ensuremath{\ensuremath{\mathit{u}}.\ensuremath{\mathit{left}} \ne nil} {\color{black} \textbf{or}} \ensuremath{\ensuremath{\mathit{u}}.\ensuremath{\mathit{right}} \ne nil} {\color{black} \textbf{do}} \\
\hspace*{1em} \hspace*{1em} \hspace*{1em} {\color{black} \textbf{if}} \ensuremath{\ensuremath{\mathit{u}}.\ensuremath{\mathit{left}} \eq nil} {\color{black} \textbf{then}} \\
\hspace*{1em} \hspace*{1em} \hspace*{1em} \hspace*{1em} \ensuremath{\mathrm{rotate\_left}(\ensuremath{\mathit{u}})}\\
\hspace*{1em} \hspace*{1em} \hspace*{1em} {\color{black} \textbf{else}} {\color{black} \textbf{if}} \ensuremath{\ensuremath{\mathit{u}}.\ensuremath{\mathit{right}} \eq \ensuremath{\mathit{nil}}}\\
\hspace*{1em} \hspace*{1em} \hspace*{1em} \hspace*{1em} \ensuremath{\mathrm{rotate\_right}(\ensuremath{\mathit{u}})}\\
\hspace*{1em} \hspace*{1em} \hspace*{1em} {\color{black} \textbf{else}} {\color{black} \textbf{if}} \ensuremath{\ensuremath{\mathit{u}}.\ensuremath{\mathit{left}}.\ensuremath{\mathit{p}} < \ensuremath{\mathit{u}}.\ensuremath{\mathit{right}}.\ensuremath{\mathit{p}}}\\
\hspace*{1em} \hspace*{1em} \hspace*{1em} \hspace*{1em} \ensuremath{\mathrm{rotate\_right}(\ensuremath{\mathit{u}})}\\
\hspace*{1em} \hspace*{1em} \hspace*{1em} {\color{black} \textbf{else}} \\
\hspace*{1em} \hspace*{1em} \hspace*{1em} \hspace*{1em} \ensuremath{\mathrm{rotate\_left}(\ensuremath{\mathit{u}})}\\
\hspace*{1em} \hspace*{1em} \hspace*{1em} {\color{black} \textbf{if}} \ensuremath{\ensuremath{\mathit{r}} \eq u} {\color{black} \textbf{then}} \\
\hspace*{1em} \hspace*{1em} \hspace*{1em} \hspace*{1em} \ensuremath{\ensuremath{\mathit{r}} \gets  \ensuremath{\ensuremath{\mathit{u}}.parent}}\\
\end{flushleft}

\end{frame}

\begin{frame}[shrink]
\frametitle{Exemplo}
\begin{figure}
  \begin{center}
  \includegraphics{../Book-datastructures/ods/latex/figs-python/treap-delete-a} 
  \end{center}
  \caption[Removendo de uma treap]{Removendo o valor 9 da Treap}
  \figlabel{treap-remove}
\end{figure}
\end{frame}

\begin{frame}[shrink]
\frametitle{Exemplo}
\begin{figure}
  \begin{center}
  \includegraphics{../Book-datastructures/ods/latex/figs-python/treap-delete-b} 
  \end{center}
  \caption[Removendo de uma treap]{Removendo o valor 9 da Treap.}
  \figlabel{treap-remove}
\end{figure}
\end{frame}

\begin{frame}[shrink]
\frametitle{Exemplo}
\begin{figure}
  \begin{center}
  \includegraphics{../Book-datastructures/ods/latex/figs-python/treap-delete-c}
  \end{center}
  \caption[Removendo de uma treap]{Removendo o valor 9 da Treap.}
  \figlabel{treap-remove}
\end{figure}
\end{frame}

\begin{frame}[shrink]
\frametitle{Exemplo}
\begin{figure}
  \begin{center}
  \includegraphics{../Book-datastructures/ods/latex/figs-python/treap-delete-d} 
  \end{center}
  \caption[Removendo de uma treap]{Removendo o valor 9 da Treap.}
  \figlabel{treap-remove}
\end{figure}
\end{frame}
\begin{frame}
\frametitle{Exercício}
Implementar uma árvore aleatória binária de busca.
\end{frame}


\begin{frame}
FIM
\end{frame}
\end{document}
